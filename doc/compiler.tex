\newcommand{\lyacc}{$\lambda$yacc\xspace}

\chapter{Compiling Theories to \lclam}\label{compiler}

%\author{Ewen Denney}

\section{Introduction}

This chapter describes the current state of the \lclam theory
compiler.  It develops the idea that a logic can be presented in a
high-level declarative style, and then {\em compiled} down into a more
procedural implementation-specific language.

The reasons for wanting to do this are:
\begin{itemize}
\item the user-unfriendliness of the current \lclam format
\item to avoid errors that can arise when entering axioms and
  inferences as low-level rewrites and methods
\item (ultimately) to aim for a logic-independent planner by factoring out
  logic-specific code
\item since a high-level representation of logics and theories, rather than
  one which is hard-wired into a low-level implementation, is easier
  to communicate.
\end{itemize}

In our case, of course, the procedural language is \lclam's format for
the paraphernalia of proof planning (expressed in HOAS) in \lprolog;
for declarative language, we simply make up our own user-friendly
syntax for declarations, inference rules, and the like.


The compiler has two modes of use: online, using the {\tt add\_theory}
command during \lclam's normal command loop, and offline, generating a
separate \lclam theory file.  The offline mode is currently deprecated
but is described here as it helps understand the general compilation
process and may prove useful in the future.

\section{Online Mode}

The command
$${\tt add\_theory} \mbox{\em ``theory-file''}$$
will read in the file
{\em theory-file}, check its well-formedness, and generate the
appropriate user-defined constants.

For example, the file {\tt simpletheory}:
\begin{verbatim}
Theory Nsa.
Use Logic arithmetic.

type real.
type hyperreal.
$
\end{verbatim}
gives the following results:
\begin{verbatim}
lclam:
add_theory "simpletheory".
Trying ...Scanning for tokens...
Tokenizer stopped after line 6
Parsing...
Successfully parsed
Checking well-formedness...
Type real
Type hyperreal
Theory is well-formed
Importing lclam declarations...
Successfully processed
Done
Done
lclam:
\end{verbatim}
We can now use the query commands to check that the definitions have been made
correctly.
\begin{verbatim}
lclam:
query_osyn (user_theory "nsa") (user_object "real") X.
default
user_theory "nsa"
real
universe
lclam:
\end{verbatim}

A more complex example of a theory file can be seen in
Section~\ref{theory}.

\section{Offline Mode}

We describe here how a user can use the files {\tt
  envgrammar.\{sig,mod\}} and\linebreak 
{\tt postprocess.\{sig,mod\}} in the
{\tt compiler} directory to parse a theory file written in concrete
syntax and generate a \lclam theory file.

We first explain how to use the compiler, and in the following
sections, explain the workings of the compiler in case the user wants
to alter it.

The phases of such a logical compilation are much the same as in the
compilation of programs: parsing of concrete into abstract syntax
(Section \ref{parsing}), a well-formedness check (Section
\ref{well-formedness}), and then code generation (Section
\ref{generation}). We give the grammar for theory files (Section
\ref{grammar}), and examples of concrete syntax (Section \ref{theory}) and
the corresponding \lclam (Section \ref{lclam}).


\subsubsection{Using the Compiler}\label{makefile}

First, edit the {\tt Makefile}, if necessary, to alter the paths appropriately.

Then, generate the parser by
\begin{alltt}
make envparser
\end{alltt}
Next, do
\begin{alltt}
make postprocess
\end{alltt}

If the theory file is {\tt theoryfile}, then the command
\begin{alltt}
parsefile "theoryfile" X.
\end{alltt}
will bind to $\tt X$ the abstract syntax that results from parsing
this file. This might be useful to you if you just want access to the
higher-order abstract syntax form of an expression.

Here, {\tt theoryfile} is the file with concrete syntax which you can,
of course, change and rename. Note that the underlying parser, \lyacc,
requires this file to terminate with a {\tt \$}.

The command
\begin{alltt}
process_file "theoryfile"
\end{alltt}
will parse the file, check its well-formedness, and output the
appropriate \lclam files. The names of the files are determined by
the line
\begin{alltt}
Theory <name>.
\end{alltt}
in the theory file, which results in {\tt <name>theory.\{sig,mod\}}. 

Finally, you then have to fit these files in with your version of
lambda clam (by setting the appropriate accumulates) and compile the
whole system. Note that the compilation files themselves should not be
compiled with \lclam.


\subsection{Parsing}\label{parsing}

The logic file is first parsed into abstract syntax. The parser is
implemented using 
\lyacc\footnote{See {\tt http://www.cs.hofstra.edu/\~{ }cscccl/parsergen/.}}, 
a parser-generator written in \lprolog, due to Chuck Liang. It
generates bottom-up {\em shift-reduce} parsers, which can be less
intuitive than top-down parsers, but are more efficient.

The user specifies their logic by writing various predicates and rules
in a \lprolog file, which accumulates the \lyacc files.

\begin{enumerate}
\item Declare grammar symbols for the abstract syntax for each nonterminal.
\item Declare the terminals.
\item Give concrete syntax for the terminals using {\tt printname}.
\item List the terminals, and nonterminals.
\item Set {\tt ntnum} to the number of nonterminals plus 1.
\item
Give the parse rules, with semantic actions to construct
the abstract syntax.

For example, rules giving the syntax for axiom declarations are
\begin{verbatim}
 rule ((syn_decl_gs D5) ==> 
        [axiom_decl_gs S7 H1 Z7]) (D5 = axiom_decl S7 H1 Z7),

 rule ((axiom_decl_gs S5 H2 Z5) ==> 
        [axiomt, axiom_name I6, qprop_gs Z3, periodt]) 
                                (S5 = I6, Z5 = Z3, H2 = []),
\end{verbatim}
The {\tt rule} construct combines a production with a semantic action.
\item
Give precedence rules.
\item Supply freshcopy clauses for non-nullary nonterminals.
\end{enumerate}


This file is compiled, and then executed. Then, a generate parser
command is called, which, assuming \lyacc accepts the file, generates
{\em another} file. This file is then compiled to give the parser, all
of which takes about a minute in total.


\subsection{Well-formedness}\label{well-formedness}

A consequence of the fact that the semantic actions attached to \lyacc
parse rules cannot contain implications (and that \lprolog does not
have assertions), is that well-formedness of the resulting abstract
syntax has to be checked in a separate pass.

However, logically inspired manipulation of terms such as this is what
\lprolog does best.  For example, well-formedness of existentials and
abstractions is checked by
\begin{verbatim}
well_formed_assertion (app exists (tuple [T, abs P])) :-
        well_formed_type T,
        pi X \ (new_const X T) => well_formed_assertion (P (user_object X)). 

well_typed_term (ty_abs T L) (T arrow U) :-
        well_formed_type T,
        pi X \ (new_const X T) => well_typed_term (L (user_object X)) U.
\end{verbatim}
Since declarations are `first-class' in \lprolog, \lprolog's context
mechanism can be used for managing declarations.

\begin{verbatim}
well_formed_decls ((inference_decl S Hyps H2 C2)::Ds) :-
        well_formed_decl (inference_decl S Hyps H2 C2),
        ((new_inference S) => well_formed_decls Ds).

well_formed_decl (inference_decl S Hyps H2 C2) :-
        not (new_inference S),
        map_pred2 group_assertions Hyps Ass,
        all_pred well_formed_assertions Ass,
        ly_append H2 [C2] Ps2,
        well_formed_assertions Ps2,
        M is "Inference " ^ S ^ "\n",
        print M.
\end{verbatim}
There are no error messages at present.

\subsection{Generation of \lclam}\label{generation}
Once the theory file has been parsed into abstract syntax and checked
for well-formedness, it is converted to \lclam format and written to a
file. Constructing the appropriate higher-order abstract syntax was
done during parsing, so this stage is mainly concerned with
abstracting over the variables, and converting propositions to
rewrites and methods.

The syntax comprises a list of declarations, each of which is
converted to the corresponding \lclam entity.

\begin{alltt}
% decl2lclam {\it +sig-file +mod-file +theory-name +logic-name +declaration}

type decl2lclam  out_stream -> out_stream -> string -> string -> syn_decl -> o.
\end{alltt}
A syntactic declaration is one of: type, typed constant, axiom,
inference, conjecture, predicate symbol, definition.  The interesting
cases are axiom and inference rules. We convert axioms to rewrite
rules, and inferences to methods.

An example of an axiom is
\begin{verbatim}
axiom ax1 [y:nat |- forall f:nat->nat. f = (x:nat \ (f y))].
\end{verbatim}
This is parsed as
\begin{verbatim}
axiom_decl "ax1" (otype_of (user_object "y") nat :: nil) 
   (app forall (tuple ((nat arrow nat) :: 
         abs (W1\ app eq 
          (tuple (W1 :: ty_abs nat (W2\ app W1 (user_object "y")) :: nil))) ::
 nil)))
\end{verbatim}
and converted to the rewrite rule
\begin{verbatim}
axiom nsa ax11 ltor (trueP) 
   (app forall 
     (tuple ((nat arrow nat) :: 
       abs (W1\ app eq (tuple (W1 :: ty_abs nat (W2\ app W1 _107597) :: nil)))
 :: nil))) 
    (trueP).
\end{verbatim}


For inference rules, the system generates preconditions to check
for required antecedents in the conclusion, and to construct the
antecedents of the hypotheses. The succedent of the conclusion is
recorded in the form of the method input, and the succedents of the
hypotheses appear in the output goal.

For example, the inference rule
\begin{verbatim}
inference and_l [G,A,B,G' |- P / G, A /\B |- P]
\end{verbatim}
parses to abstract syntax
\begin{verbatim}
(inference_decl "and_l"
[(pair [user_object "G", user_object "A", user_object "B", user_object "G'"] 
       (user_object "P"))]
[user_object "G", (app and (tuple [user_object "A", user_object "B"])), 
user_object "G'"] (user_object "P"))
\end{verbatim}
which is converted to the atomic method
\begin{verbatim}
atomic nsa and_l
        (seqGoal (H >>> (_9055)))
        (sublist (app and (tuple (_9072 :: _9088 :: nil)) :: nil) H, 
        replace_in_hyps H (app and (tuple (_9072 :: _9088 :: nil))) 
                             (_9072 :: _9088 :: nil) H1)
        true
        seqGoal (H1 >>> _9055)
        notacticyet.
\end{verbatim}
The inference rule
\begin{verbatim}
inference and_r [G |- A ; G |- B / G |- A /\B]
\end{verbatim}
parses to abstract syntax
\begin{verbatim}
(inference_decl "and_r"
[(pair [user_object "G"] (user_object "A")),
(pair [user_object "G"] (user_object "B"))]
[user_object "G"] (app and (tuple [(user_object "A"), (user_object "B")]))
)
\end{verbatim}
and converts to
\begin{verbatim}
atomic nsa and_r
        (seqGoal (H >>> (app and (tuple (_5035 :: _5052 :: nil)))))
        (H =  H1, H =  H2)
        true
        seqGoal (H1 >>> _5035) ** seqGoal (H2 >>> _5052)
        notacticyet.
\end{verbatim}


\subsection{Theory Grammar}\label{grammar}

Identifiers can contain underscores but not hyphens. A theory begins
with a declaration of its name, and the logic it uses. The logic name
is not used, but must be entered for syntactic correctness.

\begin{verbatim}
Theory-declaration ::= 
        'Theory' Id '.'
        'Use Logic' Id '.'
        (Declaration '.')*

Declaration ::= Constant-declaration
                | Type-declaration
                | Axiom-declaration
                | Inference-declaration
                | Conjecture-declaration
                | Predicate-declaration
                | Definition


Constant-declaration ::= 'const' Id Type

Type-declaration ::= 'type' Id

Axiom-declaration ::= 'axiom' Id Prop | 'axiom' Id '[' Sequent ']' 

Inference-declaration ::= 'inference' Id '[' Sequents '/' Sequent ']'

Conjecture-declaration ::= 'conjecture' Id '[' Sequent ']'

Predicate-declaration ::= 'predicate' Id Type+

Definition ::= 'define' 'const' Id ':' Type '=' Term
             | 'define' 'type' Id '=' Type

Sequents ::= (Sequent ';')* Sequent

Sequent ::= Assumptions '|-' Prop

Assumptions ::= (Assumption ',')* Assumption

Assumption ::= Prop | Id ':' Type

Prop ::= 'forall' Id ':' Type '.' Prop
       | 'exists' Id ':' Type '.' Prop
       | Prop '/\' Prop
       | Prop '\/' Prop
       | Prop '=>' Prop
       | '<' Term '=' Term '>'
       | '(' Prop ')'
       | Id
       | Id '{' Term '}'

Type ::= nat | bool | '(' Type ')' | Type '->' Type | Type '#' Type | Id

Term ::= Id
       | (Id ':' type '\' Term ')'
       | '<' Term Term '>'
\end{verbatim}

\subsection{Theory file}\label{theory}
\begin{verbatim}
Theory Nsa.
Use Logic Thol.

type real.
type hyperreal.
const z nat.
const f nat->nat#bool.

axiom sch2 [x:nat |- <x=x>].
axiom sch3 [x:nat, y:nat |- <x=y>].
axiom sch4 [x:nat, y:nat, <x=y> |- <y=x>].
axiom ax1 [y:nat |- forall f:nat->nat. <f = (x:nat \ <f y>)>].
axiom exax exists x:nat. <x=x>.
axiom badax [true, x:nat |- <x=x>].

conjecture conj1 [x:nat |- forall y:nat . <x=y>].
inference inf1 [true |- false / true, x:nat |- <x=x>].

axiom qs (forall x:nat. (exists y:nat. <x=y>)) => true.

conjecture conj2 [x:hyperreal |- exists y:hyperreal . <x=y>].

predicate even : nat.
predicate close_to : hyperreal, real.

inference easy_inf [true|-true; false|-false / x:nat |- <x=x>].

axiom pred_in_prop even{z}.

define type real_pair = real#real.

define const n:nat#bool = <f z>.

define const m:nat = z.

axiom false_e [G1, false, G2 |- P].

inference and_r [G |- A; G |- B / G |- A /\ B].

inference and_l [G1,A,B,G2 |- P / G1, A /\B,G2 |- P].

inference or_l [G,A |- P; G,B |- P / G, A\/B |- P].
\end{verbatim}

\subsection{Generated \lclam}\label{lclam}
\begin{verbatim}
module nsatheory.

all_pred P nil.
all_pred P (H::T) :- P H, all_pred P T.
sublist Xs Ys :- all_pred (X\ (member X Ys)) Xs.

has_otype nsa real universe.
has_otype nsa hyperreal universe.
has_otype nsa z (nat).
has_otype nsa f (nat arrow tuple_type (nat :: bool :: nil)).
axiom nsa sch21 ltor (trueP) (_135695) (_135695).

axiom nsa sch22 ltor (trueP) (app eq (tuple (_137129 :: _137129 :: nil))) 
(trueP).

axiom nsa sch31 ltor (trueP) (_137732) (_137749).

axiom nsa sch41 ltor (app eq (tuple (_139400 :: _139383 :: nil))) (_139383) 
(_139400).

axiom nsa sch42 ltor (app eq (tuple (_144856 :: _144839 :: nil))) (app eq 
(tuple (_144839 :: _144856 :: nil))) (trueP).

axiom nsa ax11 ltor (trueP) (app forall (tuple ((nat arrow nat) :: abs (W1\ 
app eq (tuple (W1 :: ty_abs nat (W2\ app W1 _147505) :: nil))) :: nil))) 
(trueP).

axiom nsa exax1 ltor (trueP) (app exists (tuple (nat :: abs (W1\ app eq (tuple
 (W1 :: W1 :: nil))) :: nil))) (trueP).

axiom nsa badax1 ltor (trueP) (_148474) (_148474).

axiom nsa badax2 ltor (trueP) (app eq (tuple (_149926 :: _149926 :: nil))) 
(trueP).

top_goal nsa conj1 [(otype_of (user_object "x") nat)] (app forall (tuple (nat 
:: abs (W1\ app eq (tuple (user_object "x" :: W1 :: nil))) :: nil))).
atomic nsa inf1
        (seqGoal (H >>> (app eq (tuple (_152632 :: _152632 :: nil)))))
        (sublist (trueP :: otype_of (user_object "x") nat :: nil) H)
        true
        (seqGoal (H1 >>> falseP))
        notacticyet.

axiom nsa qs1 ltor (trueP) (trueP) (app forall (tuple (nat :: abs (W1\ app 
exists (tuple (nat :: abs (W2\ app eq (tuple (W1 :: W2 :: nil))) :: nil))) :: 
nil))).

axiom nsa qs2 ltor (trueP) (app imp (tuple (app forall (tuple (nat :: abs (W1\
 app exists (tuple (nat :: abs (W2\ app eq (tuple (W1 :: W2 :: nil))) :: 
nil))) :: nil)) :: trueP :: nil))) (trueP).

top_goal nsa conj2 [(otype_of (user_object "x") (user_object "hyperreal"))] 
(app exists (tuple (user_object "hyperreal" :: abs (W1\ app eq (tuple 
(user_object "x" :: W1 :: nil))) :: nil))).
atomic nsa easy_inf
        (seqGoal (H >>> (app eq (tuple (_156487 :: _156487 :: nil)))))
        (sublist (otype_of (user_object "x") nat :: nil) H, replace_in_hyps H 
(otype_of (user_object "x") nat) (falseP :: nil) H1, replace_in_hyps H 
(otype_of (user_object "x") nat) (trueP :: nil) H2)
        true
        (seqGoal (H1 >>> trueP) ** seqGoal (H2 >>> falseP))
        notacticyet.

axiom nsa pred_in_prop1 ltor (trueP) (app (user_object "even") (user_object 
"z")) (trueP).

definition nsa real_pair0 trueP (tuple_type (user_object "real" :: user_object
 "real" :: nil)) real_pair.
definition nsa n0 trueP (app (user_object "f") (user_object "z")) n.
definition nsa m0 trueP (user_object "z") m.
atomic nsa false_e
        (seqGoal (H >>> (_25170637)))
        (sublist (falseP :: nil) H)
        true
        trueGoal
        notacticyet.

atomic nsa and_r
        (seqGoal (H >>> (app and (tuple (_162845 :: _162862 :: nil)))))
        (true, H =  H1, H =  H2)
        true
        (seqGoal (H1 >>> _162845) ** seqGoal (H2 >>> _162862))
        notacticyet.

atomic nsa and_l
        (seqGoal (H >>> (_186518)))
        (sublist (app and (tuple (_186535 :: _186551 :: nil)) :: nil) H, 
replace_in_hyps H (app and (tuple (_186535 :: _186551 :: nil))) (_186567 :: 
_186535 :: _186551 :: _186583 :: nil) H1)
        true
        (seqGoal (H1 >>> _186518))
        notacticyet.

atomic nsa or_l
        (seqGoal (H >>> (_197214)))
        (sublist (app or (tuple (_197231 :: _197247 :: nil)) :: nil) H, 
replace_in_hyps H (app or (tuple (_197231 :: _197247 :: nil))) (_197247 :: 
nil) H1, replace_in_hyps H (app or (tuple (_197231 :: _197247 :: nil))) 
(_197231 :: nil) H2)
        true
        (seqGoal (H1 >>> _197214) ** seqGoal (H2 >>> _197214))
        notacticyet.

end

\end{verbatim}


