\chapter{Working with \lclam\ within the Dream Group: The CVS Server}

\section{Introduction}
We have had a serious problem in the past that each member of the
Dream Group essentially worked on and developed their own version of
\clam.  In this way interesting and important additions, such as
critics or relational rippling, never found their way into the main
distribution.

We are anxious that a similar situation doesn't arise with \lclam\ and 
this guide is intended to set out some procedures for preventing this.

\section{Trying Out the System, Running Case Studies with existing
  Methods and Strategies}

It should be possible to run \lclam\ on your own theorems without having a
personal version of the system or compiling any \lprolog\ files
yourself.  This will done using \lclam's library
mechanism\footnote{Not yet implemented} or the command line options
for adding theory information discussed in chapter~\ref{basic}.  If at
all possible we would like to 
encourage users to work with \lclam\ through this mechanism.  In this
way any work they do consists solely of new library files which can be 
easily included in the distribution.  This work can be done using the
centrally installed executable for the latest version of
\lclam\footnote{There is currently no centrally installed version but
  there will be for version 0}.  

This method of working should be sufficient for anyone
experimenting with the existing capabilities of the system to see what 
it can do.

\section{Developing Complex Methods, Heuristics and Theories}

To customise \lclam\ with your own predicates you will need
a copy of the source code in your own directory space.  This doesn't
mean, however, that you have to work on it in isolation or that you
should make changes to the existing files in the distribution.

Firstly, unless you do so in consultation with one of the \lclam\
developers\footnote{One of the two RAs on the project, the computing
  officer or Alan Smaill}, you should not edit any files in the distribution unless it 
is a Makefile\index{Makefile}.  In this way all 
modifications made by individuals should appear in separate files
which can be imported wholesale into the main distribution without
hundreds of minor edits being needed throughout the system.  

If you anticipate that you will be working with \lclam\ in this way
for a significant period of time then you should seriously consider
getting hold of the source code through CVS\index{cvs}.  This means
that you can 
keep track of any changes being made by the developers on a regular
basis so that at the end of your work your new code still works with
any changes that may have been made with \lclam\ in the meantime.
This stops you ending up with files that only work with a now defunct
version of \lclam.  There are a number of CVS guides around which
will tell you how to use it, including a quick crib at the end of this 
appendix.  The important thing for work of this
nature is that you don't necessarily have to include your personal
work in the main source directories that everyone sees until you feel
ready to do so.  You can however update your \lclam\ source files
regularly from the main directory so that you keep track of any
changes other people are making.  Regular updating by users and
hence regular bug reporting also allows the developers to identify
quickly if they have made a change which has introduced bugs.

\section{Developing the Core of \lclam} 
There shouldn't be a large number of people who have reason to change
the core \lclam\ files.  It should mainly be the two RAs, Alan Smaill
and Gordon Reid, plus people working on the induction competition etc.
Developers should try to update and commit their work regularly.  We
should probably also aim for something like 6 monthly releases of the
code so that users not working on code in the CVS repository still
working on pretty much up-to-date version of \lclam.

\section{Undergraduates, MSc Students, Researchers visiting for less
  than 6 months}

Hopefully many of people on short projects will be able to work with
library files alone, in which case they need never worry about
acquiring a copy of the code for themselves.

It is unreasonable to ask someone who
will only work 
with the system for a few months to get to grips with CVS, nor will 
they want to cope with the vagaries of constantly changing code.  

If
you fall into this category and you
need to write files for compilation you should 
take a snapshot of the most recent version of \lclam\ in the directory in
{\tt /usr/local/reason/dream/meta-level/lambda-clam/} and work with
that.  Please bear in mind the notes above about only altering
specific files without consulting the developers.  

At the end of the project get in touch with either of the two RAs
about merging your work back into \lclam. 

\section{PhD Students, Researchers on long term projects or those
  wanting to work with the most up to date version of the system}

If you are a PhD student or a researcher with a long term project
(not counting those who can manage with 
library files alone) then you should either ensure that you take a new 
snapshot of the code every time there is a release or you should work
with the CVS repository.  While this second option makes you more
vulnerable to  
changes in the code it makes it much more
likely, at the end of your research, that the work you have done will be 
consistent with the the current version of \lclam\ and that \lclam\ will be
consistent with your work.  This makes it much more likely that your
work will get out and be used by other people.

Working with the current version in CVS also means that you can get
specific bug fixes within hours of them being made instead of waiting
for the next release.  In some cases this means you will be able to
receive bug fixes in a very short space of time after first
discovering the bug.

If using CVS, there is no need to put your work into the main
distribution until you  
are ready.  All that is required is that you regularly (say
once a month) update the files in your directory
(using {\tt cvs update -d} - see appropriate CVS manual).  This means
that any changes the developers have made in the meantime will
propagate themselves into the code you are using.  At the end of your
time using \lclam\ you simply type {\tt cvs commit} to put your work
into the main distribution for others to see and use.  

We realise this is not an ideal situation.  We would like to get to a
situation where anyone developing strategies, methods, heuristics and
theories can work through library files alone from a centrally
installed version of \lclam.  At the moment this simply isn't
possible.  While this system will undoubtedly cause many short term
headaches we hope that in the long run it will mean less work for all
concerned.


\section{CVS Crib}

CVS works on the principal of editing files in your own private
    working directory and then merging them back into a central
    repository.  This is done by the use of the commands {\tt cvs
      commit} and {\tt cvs update}.  {\tt cvs
      commit} commits any changes you have made in your working
    directory back into the main tree, while {\tt cvs update}
    merges any changes in the main tree into the files in your working 
    directory.  There are subsidiary commands to add and remove files
    from the CVS repository and to acquire a working directory (these
    are outlined in the crib below).


CVS commands generally work on everything in the directory where
      they are invoked and all its subdirectories.  This can seem a
      little strange at first.  E.g. typing {\tt cvs update} in 
      some sub-directory of the module will update all the files in
      that directory and its subdirectories, but will not update
      anything in any higher-level directories. 

To use CVS you need to set the enviroment variable \verb+CVSROOT+.
For local Edinburgh this should be set to {\tt
  /usr/local/reason/dream-src}.


\subsection{Important CVS Commands and Flags}

\subsubsection{Acquiring a Working Directory}
{\tt cvs get} will copy a snapshot 
    of the current state of the program in the repository into your
    filespace.  For example
\begin{verbatim}
cvs get lambda-clam-teyjus
\end{verbatim}
 will copy
    into your directory the current working version of \lclam\ on
    the main branch.

Note that {\tt get} is used to create directories.   The
          top-level  directory  created  is  always  added to the
          directory where checkout is invoked,  and  usually  has
          the  same name as the specified module.  

If you do not intend to use cvs for working on your files.
      I.e. you don't intend to update or commit but will simply email
      your changes at the end of your time working on them to the
      developers or you do not want your changes to get into the main
      distribution, then you can acquire a snapshot of the working
      directory using {\tt cvs export}.  This will not allow
      you to use commit or update.

If you want to 
      abandon a working directory for whatever
      reason and no longer use it for committing or updating you should 
      type {\tt cvs release .} in the directory you intend to abandon.

Members of the Dream Group who are not based at
  Edinburgh University can
get a cope of \lclam\ from the Edinburgh CVS repository by typing

\begin{verbatim}
cvs -d :pserver:anonymous@broom.dai.ed.ac.uk:/usr/local/reason/dream-src 
 login

cvs -d :pserver:anonymous@broom.dai.ed.ac.uk:/usr/local/reason/dream-src 
 checkout lambda-clam-teyjus
\end{verbatim}

The first of these will prompt you for a user name and password.  The
password is currently empty (I'm sure Gordon will change this soon
though!!)

You will not be able to commit changes this way but will be able to
track updates.


\subsubsection{Committing Changes}
Use {\tt cvs commit} when you want to     incorporate  changes
          from  your working source files into the general source
          repository.  {\tt cvs commit} will work recursively
    through all the files and subdirectories from the one in which it
    is typed.  It will only commit changes in files and directories
    which {\em already exist} in the source repository.  New files
    and directories should be added first by {\tt cvs add}.
    Similarly files and directories you have removed should be removed 
    by {\tt cvs remove}.

{\tt cvs commit} verifies that your files are up to date.
      I.e. that no one else has committed any changes since you last
      updated.  If they are not up to date it will notify you and then 
      exit without committing.  You will need to do an update before
      you can commit your files.

When all is well, an editor is invoked to allow you  to
          enter a log message that will be written to one or more
          logging programs and placed in  the  source  repository
          file.  You can control the editor used by setting the
      environment variable {\tt CVSEDITOR} or invoking
      {\tt cvs commit} with the flag {\tt -e editor}.

\subsubsection{Updating your working Directory}
 After you've run checkout to create your  private  copy
          of  the source from the common repository, other developers
          will continue changing the central source.   From  time
          to time, when it is convenient in your development process,
    you can use the {\tt cvs update -d} command from  within  your
          working directory to reconcile your work with any revisions
    applied to  the source repository since your last 
          checkout or update.

The {\tt -d}  option to creates (or deletes)
any directories
      that  exist 
          in  the  repository if they're missing from the working
          directory.  (Normally, update acts only on  directories
          that  were already enrolled in your working
          directory.)


\subsubsection{Adding a file to the Repository}
Use {\tt cvs add} to create a new file  or   directory
          in  the  source  repository.   The files or directories
          specified with add must already exist  in  the  current
          directory  (which  must  have  been  created  with  the
          checkout  command). 

For example
\begin{verbatim}
cvs add mytheory.mod
\end{verbatim} schedules your
    file {\tt mytheory.mod} to be placed in the repository
    next time you commit some changes.  After this, anyone typing
    {\tt cvs update} will get a copy of your file.  Similarly if you have
    added a new directory you can add it in the same way.

NB.  The file will not appear in the source repository until you
      commit it.

\subsubsection{Removing a file from the Repository}
Use {\tt cvs remove} to remove a file you have deleted from the 
    source repository.  
As with {\tt cvs add} The files are not actually removed until
    you apply your changes  to  the repository with commit.  

\subsection{Final Word}
This is only an overview of the major aspects of working with CVS.
        You can find much more detail in the man pages or in various
        guides and textbooks.