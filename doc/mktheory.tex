\chapter{Creating Theories as Library Files\index{library file} or
\lprolog\ files\index{theory file}}
\label{mktheory}

\section{Introduction}
At present, the best way to add your own goals\index{goal},
methods\index{methods}, rewrite rules\index{rewrite rule}, etc. to
\lclam\ is to compile a theory file into its bytecode.  This chapter
aims to overview the main components of a theory file and discuss how
to compile and debug it.

\section{Theory File Components}
\subsection{Theory Name\index{theory name}}
In most cases it is advisable to name your theory by introducing a
constant of type {\tt theory}\index{theory type} in your theory's
signature file\index{signature file}.  

\subsection{Syntax Constants\index{syntax constants}}
NOTE:  We have a utility for writing syntax in a more natural format
than that required by \lclam's internals.  See chapter~\ref{compiler}.

Syntax constants can be any function symbols or constants that you
wish to include in your theory.  These should be declared in the
signature file of your theory (as items of type {\tt osyn} \index{osyn
  type}).  You should also include a clause of the predicate {\tt
  has\_otype}\index{has\_otype} in the module file for your theory.
{\tt has\_otype} takes as arguments the name of the theory, the name
of the constant and the type.  New types should be declared using {\tt
  is\_otype}\index{is\_otype}.  {\tt is\_otype} takes the name of the
theory, the new type and two lists of constructors one for ``base''
elements of the type and one for ``constructor'' functions.  Types are
given the type {\tt universe}\index{universe}.  For example the
following is taken from the arithmetic theory\index{arithmetic theory}
and declares the type {\tt nat}\index{nat} with the constructors {\tt
  zero}\index{zero} and {\tt s}\index{a} and the function {\tt
  plus}\index{plus}.

\begin{verbatim}
is_otype arithmetic nat [zero] [s].

has_otype arithmetic zero nat.
has_otype arithmetic s (nat arrow nat).
has_otype arithmetic plus ((tuple_type [nat, nat]) arrow nat).
\end{verbatim}

The signature file for the theory contains the declarations

\begin{verbatim}
type    nat     osyn.

type    zero    osyn.
type    s       osyn.
type    plus    osyn.
\end{verbatim}


\subsection{Goals\index{goal}}

New goals are declared using the predicate {\tt
  top\_goal}\index{top\_goal}.  This takes four arguments: the theory
name; the name of the goal; a list of hypotheses; and finally a goal
conclusion.  So, for example, again from arithmetic\index{arithmetic
  theory}, we have in the signature

\begin{verbatim}
type assp           query.
\end{verbatim}

and in the module

\begin{verbatim}
top_goal arithmetic assp []
    (app forall (tuple [nat, 
     (abs z\ (app forall (tuple [nat, 
      (abs y\ (app forall (tuple [nat, 
       (abs x\ (app eq (tuple [
        (app plus (tuple [
         (app plus (tuple [x, y])), z])), 
        (app plus (tuple [x, 
         (app plus (tuple [y, z]))]))])))])))])))])).
\end{verbatim}

\subsection{Atomic Methods}

Atomic methods\index{method!atomic} are declared by writing clauses
for the predicate: \\
{\tt atomic: theory -> meth -> goal -> o -> o -> goal -> tactic -> o.} \\
It takes as arguments the name of the theory file\index{theory file},
the name of the method\index{method!name}, the input
goal\index{goal!input}, the preconditions\index{preconditions} and
postconditions\index{postconditions} (written in \lprolog) the output
goal\index{goal!output}, and the corresponding tactic\index{tactic}
(at present this is always the placeholder {\tt
  notacticyet}\index{notacticyet}).  Most internal goals will be
sequent goals\index{sequent goal} created using the {\tt
  seqGoal}\index{seqGoal} constructor: {\tt seqGoal (H >>> G)}.

If the method will instantiate some variable (input as part of the
method name) (e.g. rewriting\index{rewriting} methods frequently
instantiate a variable with the name of the rewrite rule\index{rewrite
  rule} used) then an additional predicate {\tt
  change\_method\_vars}\index{change\_method\_vars} must also be used.
This simply relates the method name to itself but without enforcing
unification of any variables that will need instantiating.  This
predicate is used internally to ensure that the variable is not
accidentally instantiated for subsequent method calls during planning.

For instance, the wave method\index{wave method} is declared in
the wave theory\index{wave theory} files.  In the signature we have:

\begin{verbatim}
type wave_method rewrite_rule ->  meth.
\end{verbatim}

and the module itself contains the method definition:

\begin{verbatim}
atomic wave (wave_method Rule)
        (ripGoal (Hyps >>> Goal) Skel E1)
        (ripple_rewrite Rule Skel (Goal @@ E1) (NewGoal @@ E2) Cond,
         (embeds_list E2 Skel NewGoal,
         (measure_check red Dir E2 E1 nil Embedding,
         (for_all Embedding sinkable,
         (trivial_condition Cond Hyps)))))
        true
        (ripGoal (Hyps >>> NewGoal) Skel Embedding)
        notacticyet.
change_method_vars (wave_method Rule) (wave_method _) :- !.
\end{verbatim}

\subsection{Compound Methods\index{method!compound}}
Compound methods are declared by writing clauses for the predicate: \\
{\tt compound: theory -> meth -> meth -> (list int) -> o -> o.}\\
It takes as arguments the name of the theory \index{theory file}, the
name of the method\index{method!name}, the methodical
expression\index{methodical!expression} which makes up the method, the
address where the method was applied (typically blank) and
preconditions\index{preconditions} (in \lprolog).

If the method will instantiate some variable then, as with atomic
methods, the predicate {\tt
  change\_method\_vars}\index{change\_method\_vars} will have to be
extended.

So, for instance, a simplfied version of the step case method\index{step case method}\footnote{The actual {\tt step\_case} method used in \lclam\ is more complex than this in order to deal with a number of special cases.} in the 
induction theory\index{induction theory} could be declared in the signature
file as  

\begin{verbatim}
type   step_case  meth.  
\end{verbatim}

and in the module as

\begin{verbatim}
compound induction step_case
        ( cut_meth
           ( then_meth set_up_ripple 
               (then_meth (repeat_meth (wave_method R))
                  (then_meth post_ripple
                        (try_meth fertilise                                 
                         )
                  )
               )
            )
         )
        _
        true.
\end{verbatim}

Compound methods\index{method!compound} use
methodicals\index{methodical} to build up more complex proof
strategies from other methods.  The methodicals available in \lclam\ 
are listed below.
\begin{center}
\begin{tabular}{|l|l|} \hline
Methodical & Type \\ \hline
id\_meth\index{id\_meth} & {\tt meth} \\ 
triv\_meth\index{triv\_meth} & {\tt meth} \\
orelse\_meth\index{orelse\_meth} & {\tt meth -> meth -> meth} \\
cond\_meth\index{cond\_meth} & {\tt (goal -> o) -> meth -> meth ->
meth} \\
try\_meth\index{try\_meth} & {\tt meth -> meth} \\
repeat\_meth\index{repeat\_meth} & {\tt meth -> meth} \\
then\_meth\index{then\_meth} & {\tt meth -> meth} \\
then\_meths\index{then\_meths} & {\tt meth -> meth -> meth} \\
map\_meth\index{map\_meth} & {\tt meth -> meth} \\
complete\_meth\index{complete\_meth} & {\tt meth -> meth} \\
cut\_meth\index{cut\_meth} & {\tt meth -> meth} \\
pair\_meth\index{pair\_meth} & {\tt meth -> meth -> meth} \\ 
best\_meth\index{best\_meth} & {\tt list meth -> meth} \\ \hline
\end{tabular}
\end{center}

A brief description of the methodicals\index{methodical} follows:
\begin{description}
\item[id\_meth\index{id\_meth}] Automatically succeeds.  This is used
  in the implementation of methods such as {\tt
    try\_meth}\index{try\_meth} but can also be useful when dealing
  with more than one subgoal.  For instance the {\tt
    induction\_meth}\index{induction\_meth} returns two subgoals a
  step case\index{step case} and a base case\index{base case}.  The
  step case is given to the {\tt step\_case}\index{step\_case} method
  while the base case is to be treated like most other goals by having 
  the default {\tt induction\_top\_meth}\index{induction\_top\_meth}
  applied to it.  The method, {\tt ind\_strat}\index{ind\_strat}, does 
  this as follows:
\begin{verbatim}
compound induction (ind_strat IndType)
      ( then_meths (induction_meth IndType Scheme Subst)
                   (pair_meth (map_meth (step_case IndType)) (map_meth id_meth)) )
        _
        true.
\end{verbatim}
This applies the {\tt induction\_meth} and then applies {\tt
  step\_case} method to the first subgoal (the step case) and {\tt
  id\_meth} to the second subgoal (the base case).  Since {\tt
  id\_meth} does nothing the result is subgoal is unchanged an is
passed as a result of {\tt ind\_strat} to the top level repeat.
\item[triv\_meth\index{triv\_meth}] Only succeeds if the goal is {\tt
    trueGoal}.  Is used in the implementation of {\tt
    complete\_meth}\index{complete\_meth}.
\item[orelse\_meth\index{orelse\_meth}] Succeeds if one of the two
  methods succeeds.
\item[cond\_meth\index{cond\_meth}] If the condition evaluates to true 
  then it tries the first method otherwise it tries the second.
\item[try\_meth\index{try\_meth}] Attempts the listed method but does
  not fail if the method fails.
\item[repeat\_meth\index{repeat\_meth}] Attempt the method as many
  times as possible.  The method must succeed once.
\item[then\_meth\index{then\_meth}] Tries the first method and then
  apples the second method to the result of the first.
\item[then\_meths\index{then\_meths}] In the current implementation
  this is identical to {\tt then\_meth}\index{then\_meth} but the
  intention is that it should provide finer control over multiple
  subgoals while {\tt then\_meth} should be more brute force.
\item[map\_meth\index{map\_meth}] Maps the method does a list of
  subgoals.
\item[complete\_meth\index{complete\_meth}] Succeeds only if the
  output of the method is {\tt trueGoal}\index{trueGoal}.
\item[cut\_meth\index{cut\_meth}] Indicates that if the method
  succeeds then if the proof subsequently fails alternative successes
  of the method should not be considered.  This is not implemented in
  the current version of \lclam.  This is intended to be like prolog cut.
\item[pair\_meth\index{pair\_meth}] Applies to two subgoals and
  applies the first method to the first subgoal and the second method
  to the second subgoal.
\item[best\_meth\index{best\_meth}] Takes a list of methods (each of
  which may be either atomic or compound), computes a heuristic score
  for each method using the current subgoal, and applies the method
  with the lowest score to the subgoal.  If that fails, the method
  with the next lowest score is attempted, and so on.  The heuristic
  scoring predicate is \texttt{score\_method: goal -> meth -> int ->
    o.}  To write a scoring function for a method, one just writes a
  clause for this predicate along with the method definition.  The
  file theories/best\_first\_test.mod contains an example best-first
  top method and some fixed method scores (which the user should feel
  free to experiment with).
\end{description}

Methodicals\index{methodicals} can thus be used to create complex
strategies tailored to attempting specific sorts of problem.  More and 
more complex strategies can be created out of the blocks of simpler
strategies.  It is such a strategy that is applied to a goal as its
``top method''.  

\section{Critics\index{critics}}

The critics code is still in development and has yet to be properly
documented.  The modules {\tt core\_theories/wave\_critics.mod} and
{\tt theories/ireland\_critics.mod} contain the currently implemented
critics and we advise the interested reader to consult these files.

Critics are invoked when a method within a {\tt
  patch\_meth}\index{patch\_meth} fails.  {\tt patch\_meth} takes two
arguments, a method and a critic, and applies the critic to the goal
when the method fails.  As a side effect it also prints out the
failing methods preconditions --- this can be used for debugging by
replacing the appearance of the method in your strategy with {\tt
  patch\_meth Method id\_crit}\index{id\_crit}.

There are four built-in basic critics\index{critic!built-in} that can
be used to build up critic strategies:

\begin{verbatim}
type    pop_critic      (list int) -> crit.
type    open_node       crit.
type    continue_crit   (meth -> meth) -> crit.
type    roll_back_to_crit       meth -> (list int) -> crit.
\end{verbatim}

{\tt pop\_critic}\index{pop\_critic} returns the address of the top
node on the agenda\index{agenda}.  {\tt open\_node}\index{open\_node}
makes the node to which the critic is currently applying open and {\tt
  continue\_crit}\index{continue\_crit} transforms the methodical
expression at a node as specified.  {\tt
  roll\_back\_to\_crit}\index{roll\_back\_to\_crit} deletes all nodes
in the plan after the last occurence of the specified method and
places the node at which the method was applied at the top of the
agenda\index{agenda}.

Compound critics\index{critic!compound} are glued together using {\em
  criticals}\index{criticals}, which are analogous to methodicals for
lguing together methods.  \lclam's criticals are shown in the
following table.
\begin{center}
\begin{tabular}{|l|l|} \hline
Critical & Type \\ \hline
fail\_crit\index{fail\_crit} & {\tt crit} \\
id\_crit\index{id\_crit} & {\tt crit} \\
orelse\_crit\index{orelse\_crit} & {\tt crit -> crit -> crit} \\
cond\_crit\index{cond\_crit} & {\tt    (plan\_state -> o) -> crit -> crit -> crit.} \\
try\_crit\index{try\_crit} & {\tt         crit -> crit.} \\
repeat\_crit\index{repeat\_crit} & {\tt      crit -> crit.} \\
then\_crit\index{then\_crit} & {\tt        crit -> crit -> crit.} \\
subplan\_crit\index{subplan\_crit} & {\tt     (list int) -> crit -> crit.} \\
some\_crit\index{some\_crit} & {\tt        (\_ -> crit) -> crit.} \\ \hline
\end{tabular}
\end{center}

Most of these criticals\index{critical} work analogously to the
methodical\index{methodical} with the associated name.  However, {\tt
  subplan\_crit}\index{subplan\_crit} and {\tt
  some\_crit}\index{some\_crit} require explanation.  {\tt
  subplan\_crit} apples the critic to a particular address in the plan
-- these need not be the address at which a method failed.  {\tt
  some\_crit} takes a function which produces a critic as its output.
This is then applied to a meta-variable and the resulting critic
applied -- {\em to be honest I'm not clear what function this is
  supposed to serve - Louise}.  There is more discussion of critics in
chapter~\ref{core}.

WARNING: in Teyjus, the comma used to sequence predicates
associates to the left.  The way the critics planner works means that
you will have to explicitly bracket your method
preconditions\index{method!preconditions} in order to get the planner
to consider them one at a time (as shown in the wave method\index{wave
  method} above).

\section{Compiling your Theory\index{theories!compilation}}

Say your theory is called {\tt mytheory} and is in the files {\tt
  mytheory.sig} and {\tt mytheory.mod}.  

If your theory depends on no other theories add

{\tt accum\_sig lclam.} 
At the top of your theory's {\tt .sig} file and

{\tt accumulate lclam.} 
At the top of your theory's {\tt .mod} file.

It your theory depends upon other theories then you must should
replace the {\tt lclam} accumulation headers with the name of these
theories.  You should try and minimise the number of theories
accumulated into your modules and you should make sure that no theory
can be accumulated twice - e.g. by being accumulated into your module
and into one of the other theories accumulated into the module. 

The {\tt lclam} command runs \lprolog\ executable called {\tt test.lp}
in the {\tt planner/} directory.  If you want to incorporate your
theory file into those used by {\tt test.lp} then you will need to add
accumulation commands for your theory to {\tt test.sig} and {\tt
  test.mod}, and similarly if you wish to modify an existing theory to use
predicates in your theory.

Otherwise you may run your theory without all the other stuff
which is included in test.  First you must compile your theory into an 
executable.

If your theory is in the {\tt theories/} subdirectory of the
distribution.  Type {\tt make mytheory.lp} in the {\tt theories/}
subdirectory.  Otherwise you need to include the location of your
theory on your {\tt TJPATH}\index{TJPATH}, and then type {\tt tjcc
  mytheory.lp} in that directory.  

The command:

{\tt \$LCLAM\_HOME/bin/lclam mytheory}

will then act just like {\tt lclam} except that the only methods 
etc. available will be those in your theory.

\subsection{If your theory will not Compile}.
The first thing to do is to check for warnings and error messages in
the output generated by the compiler.   Many such errors will look
like the following:
\begin{verbatim}
./mytheory.mod:57.1: Error: clash in operator and operand type
\end{verbatim}
This tells you the name of the module in which the error occured, {\tt
  mytheory} in this case, the line in that module which is causing the 
error, 57 in this case, and the character in the line, 1 in this
case.  It will then give some details of the error, usually these will 
be type checking failures and the compiler will give both the expected 
type and the existing type.  

It is always worth checking for warnings even if compilation has
successfully completed.  For instance the compiler warns if it meets
an undeclared constant (i.e. one for which no type has been given).
This is often a sign of mispelling somewhere.  The compiler does not
warn about single occurences of variable names in predicate clauses -- 
unlike most prolog compilers which do warn of this.

If the compiler is no longer generating errors or warning and your
module still will not compile or run there are a number of possible
causes.  In the case of failing to compile common messages are {\tt
  Access to unmapped memory!}, {\tt Signal 10} or {\tt Signal 6}.  All
of these errors can be fixed by the addition or removal (seemingly at
random) of print statements or by commenting in or out extra
predicates which are not used in the code.  These error messages are
caused by a bug in the Teyjus compiler which hopefully will become
less and less frequent as Teyjus becomes more stable.

If your program compiles but fails to run with the message of the form
{\tt Loader error: X overflow} then it has exceeded
built-in limits in the Teyjus simulator.  It is possible to extend
these limits in the Teyjus source and then recompile, contact {\tt
  dream@dream.dai.ed.ac.uk} for details.

\section{Some Useful Built-In Predicates}
There are a number of standard list and map predicate you may want to
use in developing \lclam\ theories.  You can find this by looking at
the files in the {\tt utils/} subdirectory of the \lclam\
distribution.

Other predicate you may find useful are:

\hrule
\vspace{2mm}
\begin{Large}
\noindent{\tt obj\_atom}\
\end{Large}index{obj\_atom}
\vspace{2mm}
\hrule
\vspace{2mm}

\noindent{{\tt type obj\_atom osyn -> o.}}

\subsection*{Synopsis}
Succeeds if the input is an atomic piece of syntax (i.e. not a
tuple\index{tuple}, function application\index{function application},
$\lambda$-abstraction\index{$\lambda$-abstraction} etc.).

\vspace{2mm}
\hrule
\vspace{2mm}
\begin{Large}
\noindent{\tt subterm\_of}
\end{Large}\index{subterm\_of}
\vspace{2mm}
\hrule
\vspace{2mm}

\noindent{\tt type subterm\_of osyn -> osyn -> (list int) -> o.}

\subsection*{Synopsis}
Succeeds if the first argument is a subterm of the second, appearing
at the position in the term indicated by the third argument.

\vspace{2mm}
\hrule
\vspace{2mm}
\begin{Large}
{\tt has\_otype\_}
\end{Large}\index{has\_otype\_}
\vspace{2mm}
\hrule
\vspace{2mm}

\noindent{\tt type has\_otype\_ osyn -> osyn -> o.}

\subsection*{Synopsis}
Returns the object type of the
syntax given as its second argument.  

\vspace{2mm}
\hrule
\vspace{2mm}
\begin{Large}
\noindent{\tt env\_otype}
\end{Large}\index{env\_otype}
\vspace{2mm}
\hrule
\vspace{2mm}

\noindent{\tt type env\_otype osyn -> (list osyn) -> osyn -> o.}

\subsection*{Synopsis}
Mostly used to return (as its third argument) the object type of the
syntax given as its first argument, given an environment (list of
hypotheses, some of which may contain typing information) given as its 
second argument.

\vspace{2mm}
\hrule
\vspace{2mm}
\begin{Large}
\noindent{\tt replace\_at}
\end{Large}\index{replace\_at}
\vspace{2mm}
\hrule
\vspace{2mm}

\noindent{\tt type replace\_at osyn -> osyn -> osyn -> (list int) -> o.
}

\subsection*{Synopsis}
Third argument is the first argument with the subterm indicated by the 
fourth argument replaced by that indicated by the second argument.

\vspace{2mm}
\hrule
\vspace{2mm}
\begin{Large}
\noindent{\tt replace\_in\_hyps}
\end{Large}\index{replace\_in\_hyps}
\vspace{2mm}
\hrule
\vspace{2mm}

\noindent{\tt type replace\_in\_hyps   (list osyn)  -> osyn -> (list
  osyn) -> (list osyn) -> o.}

\subsection*{Synopsis}
Replaces one hypothesis with a list of hypotheses.  The first argument 
is the original list of hypotheses, the second is the hypothesis to be 
replaced, the third the list of hypotheses to appear in its stead and
the forth the resulting hypothesis list.  This predicate appears in the
{\tt logic\_eq\_constants} theory\index{logic\_eq\_constants} and so
this will need to be accumulated to use this predicate.

\vspace{2mm}
\hrule
\vspace{2mm}
\begin{Large}
\noindent{\tt printstdout}
\end{Large}\index{printstdout}
\vspace{2mm}
\hrule
\vspace{2mm}

\noindent{\tt type printstdout A -> o.}

\subsection*{Synopsis}
Prints its argument to standard out followed by a new line.

\vspace{2mm}
\hrule
\vspace{2mm}
\begin{Large}
\noindent{\tt testloop}
\end{Large}\index{testloop}
\vspace{2mm}
\hrule
\vspace{2mm}

\noindent{\tt type testloop o -> o.}

\subsection*{Synopsis}
Allows the interactions at the command line to be simulated.

\subsection*{Example}

The following piece of code provides a predicate which will set up a
particular contect (e.g. it uses all the induction schemes in
arithmetic and lists) and then plans a goal in that context.  This
saves the user typing in the list of instructions each time they want
to test their methods.
\begin{verbatim}
benchmark_plan test Meth Query :-
        testloop (add_theory_to_induction_scheme_list arithmetic,
        (add_theory_to_induction_scheme_list objlists,
        (set_sym_eval_list [idty, s_functional, cons_functional, leq1, leq2, leq_ss, assAnd1, prop3, prop4, prop5, prop6],
        (add_theory_defs_to_sym_eval_list arithmetic,
        (add_theory_defs_to_sym_eval_list objlists,
        (set_wave_rule_to_sym_eval,
        pds_plan Meth Query)))))).
\end{verbatim}
NOTE: bracketing of commands has had to be made explicit for this to
work.  It is possible to write your own clauses for {\tt
  benchmark\_plan}\index{benchmark\_plan} the first argument is the
name of a theory and so can be used to determine precisely which
context is wanted by the user.  Obviously it also possible to write
completely new predicates that involve {\tt testloop} to perform other 
such tasks.  Examples of {\tt benchmark\_plan} can be seen in the {\tt 
  list\_benchmarks} theory\index{list\_benchmarks}, the {\tt
  ordinal}\index{ordinal} theory and a number of others.

\subsection{List and Mapping Utilities}
A number of utility predicates for manipulating lists and mapping
predicates down lists have been implemented in \lclam\ on top of those 
already existing in Teyjus.  These can be found in the files {\tt
  lclam\_list} and {\tt
  lclam\_map}\index{lclam\_list}\index{lclam\_map} in the {\tt util/}
subdirectory of the distribution.

\section{Debugging Theories\index{debugging}}

It is described above how {\tt patch\_meth}\index{patch\_meth} can be
used to provide information about the failure of a methods
preconditions.  There is also a predicate that can be embedded into
the code to print out information on backtracking:


\vspace{2mm}
\hrule
\vspace{2mm}
\begin{Large}
\noindent{\tt on\_backtracking}
\end{Large}\index{on\_backtracking}
\vspace{2mm}
\hrule
\vspace{2mm}

\noindent{\tt 
type on\_backtracking o -> o.
}

\subsection*{Synopsis}
This can be supplied with a \lprolog instruction which it will only
execute when backtracked over.  For example, it is used in the partial 
data structure planner\index{PDS planner} to print information when a
method is being backtracked over, using the call

\begin{verbatim}
on_backtracking 
        (pprint_string "backtracking over",
        pprint_name Method),
\end{verbatim}
