\documentclass[11pt]{article}

\begin{document}

\title{$\lambda$-Clam v3.2.0: Release Note} 

\author{James Brotherston}

\date{February 28, 2002}

\maketitle

\section{New features with this release}

\subsection{Best-first methodical}
We have extended the $\lambda$-Clam methodical language with a new
methodical \verb+ best_meth: (list meth) -> meth + which takes a list
of methods (each of which may be either atomic or compound), computes
a heuristic score for each method using the current goal, and applies
the method with the lowest score.  If that fails, the method with the
next lowest score is attempted, and so on. \\

\noindent Currently no heuristic scoring functions have been written --- basic
testing has been carried out using hard-coded method scores.  The
heuristic scoring predicate is \texttt{score\_method: goal -> meth ->
  int -> o.}  To write a scoring function for a method, one just
writes a clause for this predicate along with the method definition. \\

\noindent The file theories/best\_first\_test.mod contains an example
best-first top method and some fixed method scores (which the user
should feel free to experiment with).

\subsection{Iterative deepening planner}
$lambda$-Clam now incorporates an iterative deepening planner in
addition to the standard depth-first PDS planner.  The default
configuration uses an initial depth limit of 3 and a depth increment
of 2 after each pass.  At the moment these variables are hard-coded in 
but we can include them as command line arguments in the future if
there is demand for it.  The depth of the current plan node is also
displayed in the planning output after the node address. \\

\noindent Call the iterative deepening planner with the command
\texttt{pds\_iter\_plan [method] [goal]}.  If a spypoint is set at the top
method (\texttt{set\_spypoint my\_top\_meth}) this will have the effect
of asking the user whether to continue at each new iteration.

\subsection{MathWeb integration}
Work on integrating $\lambda$-Clam with the MathWeb system has
resulted in the addition of the following new modules:

\begin{description}
\item[planner/mathweb.mod]
predicates for establishing socket communication with the MathWeb-SB and
for requesting and accessing services.

\item[theories/mw\_top.mod] new alternative top module for
  $\lambda$-Clam / MathWeb which imports all the necessary theory
  files.

\item[theories/analytica.mod]
containing the analytica examples and examples taken 
from Toby Walsh's work (Fibonacci numbers, closed forms of sums).

\item[theories/otter.mod] method that calls Otter on subgoal.

\item[theories/maple.mod] method that calls Maple simplification.
  Contains the code needed for the check whether a formula is
  arithmetic.

\end{description}

\subsection{Mutual induction module}
The new module theories/mutual\_induction.mod in the distribution
contains a method and supporting predicates for handling certain types
of mutual induction.  The module is still in development at this time
and so the use of this method is not recommended.  Normal
$\lambda$-Clam operation should not be affected by this addition.

\subsection{Reset command}
A reset command \texttt{reset\_lclam} has been added to the system.
When executed, it clears the induction schemes list, the sym\_eval
list, the general list and the wave rule list.

\section{Fixes and improvements to v3.1.0}
\subsection{Goal pretty printing}
The pretty printer code has been extended to include pretty printing
of andGoals and allGoals, which should get rid of (most of) the
\texttt{"Don't know how to display goal..."} messages.

\subsection{Compound method slots}
Compound methods now have an extra slot to compound methods which is
for the current address of the plan node, in connection with experiments
with preconditions for compound methods (in particular to prevent
repeat attempts at symbolic evaluation which is causing a lot of our
non-termination). 

\subsection{Parameterisation of induction methods}
The methods induction\_top\_meth, induction\_critics, ind\_strat\_*,
step\_case\_* have been changed so that they take an argument:
normal\_ind, with\_critics, exi.  This way instead of having a number
of methods called, say ind\_strat and ind\_strat\_critics and
ind\_strat\_exi which differed only in the step case method they
called, there is just one ind\_strat method which takes an argument.

\section{Acknowledgements}
We would like to credit the following members of the DREAM group for
all their hard work and assistance on $\lambda$-Clam v3.2.0: Louise
Dennis, Alan Smaill, and J\"urgen Zimmer.

\section{Contact}
The main $\lambda$-Clam developer is James Brotherston, email:
jjb@dai.ed.ac.uk or telephone: +44 (0)131-650-2721.

\end{document}
