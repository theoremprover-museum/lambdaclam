\documentclass[11pt]{article}

\begin{document}

\title{$\lambda$-Clam v4.0.0: Release Note} 

\author{James Brotherston}

\date{August 26, 2002}

\maketitle

\section{New features with this release}

\subsection{New system structure (Brotherston)}

The $\lambda$-Clam module structure has been entirely revamped for
this release, so that the various system components are separated out
from each other and the code is more efficient.  These changes should
be transparent to users writing their own theories; for developers,
schematics of the new module structure are in chapter 10 of the manual
(or see BBN 1436).  Some problems with the Teyjus abstract machine
have also been fixed as a result of these changes.

\subsection{QED command (Brotherston)}
Taking inspiration from the Isabelle theorem prover, we have
implemented a $\lambda$-Clam version of Isabelle's {\tt qed} command, which
converts a proven goal to a rewrite rule which can then be used in
future proofs.  The command syntax is: {\tt qed \{fwd,bwd\} [goal].}
The first argument is used to indicate whether the lemma is to be used
for forward or backward reasoning, and affects the polarity of the
rewrite rule generated.  The second argument is the goal to be used in
generating the rewrite rule.  The name of the rewrite rule generated
by {\tt qed fwd mygoal} is {\tt user\_rewrite(``mygoal\_fwd'')} (and
similarly for {\tt qed bwd}).

Unlike the Isabelle version, $\lambda$-Clam's {\tt qed} does \emph{not}
require a goal to have been proven before a rewrite can be generated
from it.  This is in order to allow users to perform `rapid
prototyping' of large proofs, deferring the proofs of sub-lemmas until
later in the development.

For more information on {\tt qed} see chapter 4 of the manual, or
refer to BBN 1433.

\subsection{GS decision procedures (Janicic)}
The {\tt src/gs} directory contains a $\lambda$-Clam implementation
of the GS framework for combining and augmenting decision procedures.
Documentation for this system is in {\tt doc/lgs-tr.ps}.

\subsection{Miscellaneous}
A problem has been fixed which was causing variables in method
continuations to become unified across goal conjunctions.  A few other
minor fixes have also been made.

\section{Contact}
The main $\lambda$-Clam developer is James Brotherston, email:
jjb@dai.ed.ac.uk or telephone: +44 (0)131-650-2721.

\end{document}
