\documentclass[11pt]{article}

\begin{document}

\title{$\lambda$-Clam v3.3.0: Release Note} 

\author{James Brotherston}

\date{July 16, 2002}

\maketitle

\section{New features with this release}

\subsection{Ireland / Walsh generalisation critics (Dennis)}
The files \texttt{whisky.\{mod,sig\}} contain work on the whisky
problem; the same whisky problem and a couple of similar ones have
gone into the benchmark set.  The set of sample theorems now includes
all those in Andrew Ireland's critics paper.  Most of these were
already in the corpus but the new ones can be found in
\texttt{ireland\_critics.\{mod,sig\}}.  The goals are rather
unhelpfully named icT6, icT7 etc. from the original paper.  

The generalisation critic now attempts the "Walsh" critic first and
then the "Ireland" critic.  Both of these critics have undergone some
debugging.  

\subsection{Two-variable induction (Dennis)}
The ability to induct on two variables at once (if this is a better
option) has been added.  For efficiency reasons, ripple analysis first
tries to induct on only one variable and only looks at two variables
if there are no unflawed schemes.  If there are no unflawed schemes on
two variables it then checks three and so on, up to the number of
forall quantifiers at the start of the conclusion of the goal.  There
are now 9 more theorems from the old clam corpus that can be proved
and these have been moved into the benchmark set.

\subsection{Specialised rewriting methods (Brotherston)}
Two new rewrite lists have been added: {\tt user\_lemma\_list} and
{\tt user\_defn\_list}.  The top-level commands \texttt
  {add\_to\_user\_\{lemma,defn\}\_list X}, \\
\texttt {delete\_from\_user\_\{lemma,defn\}\_list X}, and \texttt
  {set\_user\_\{lemma,defn\}\_list X} have been implemented; the
members of list X must be declared as lemmas / definitions
respectively in order for these commands to succeed.

\noindent Correspondingly, two new methods have been added to {\tt
  rewriting.\{mod,sig\}}:
\begin{verbatim}
deploy_lemma:  rewrite_rule -> meth
unfold_defn:   rewrite_rule -> meth
\end{verbatim}
These consult the user\_lemma\_list / user\_defn\_list respectively
and attempt to rewrite the goal once with some rule in the list.  The
reasoning behind this separation is that unfolding a definition is
typically something of a last resort, whereas it is generally
desirable to use key lemmas at the first opportunity (although of
course you can use the methods however you like).

Two other rewrite rule constructs also exist in $\lambda$-Clam: {\tt
  axiom} and {\tt hypothesis}.  Analogues of the rule lists and
methods described above do not currently exist, but would be simple to
implement if there is any demand.

\subsection{Modal properties of processes (Brotherston)}
The files {\tt
  theories/ccs\_hmf\_\{syntax,methods,theory\}.\{mod,sig\}} contain
work on a theory of properties of CCS processes expressed as modal
formulas, based on Stirling's book ``Modal and Temporal Properties of
Processes'' (Springer, 2001).  The theory is still at a
work-in-progress stage, although it does contain 10 or so examples
which plan successfully.

\subsection{Claudio planner (Brotherston)}
You can now have the plan bound to a meta-variable if you wish.  To do
this, use the command: \\

\verb+ claudio_plan method query Plan + \\
or:  \verb+ claudio_iter_plan method query Plan + \\

\noindent which invoke the standard PDS depth-first and iterative deepening
planners respectively.  The variable Plan is of type
\texttt{plan\_state} and is bound to the plan state at the end of
execution.  Note that for Teyjus to remember the variable binding you
have to write everything as one clause e.g. \\

\verb+ claudio_plan top_meth mythm Plan, printstdout Plan.+

\subsection{Improved output control (Brotherston)}
$\lambda$-Clam now has three output verbosity modes which can be toggled using the commands: \\

\verb+ verbose_mode. + / \verb+ silent_mode. + / \verb+ super_silent_mode. + \\

\noindent 
Verbose is the default output mode and prints all the planning
information.  Silent mode suppresses failed method applications and
backtracking (as before).  Super-silent mode also suppresses the
printing of the goal (but not the address), which is very useful for
debugging plans in which the goals become unmanageably large.  


\section{Fixes and improvements to v3.2.0}

\subsection{Rewrite list commands}
All the top-level commands of the form {\tt add\_list\_to\_X} and {\tt
  delete\_list\_from\_X} have been renamed to {\tt add\_to\_X} and
{\tt delete\_from\_X}.  The old versions of {\tt add\_to\_X} and {\tt
  delete\_from\_X} (which added and deleted single items) no longer
exist, so all manipulation of rewrite rule lists is done in a
consistent way.  For example:
\begin{verbatim}
add_to_sym_eval_list [myrewrite].
add_to_sym_eval_list [rule1,rule2,rule3].
delete_from_sym_eval_list [rule2].
\end{verbatim}

\subsection{MATHWEB setup \& documentation (Zimmer)}
The manual now contains a revised chapter on how to set up and use the
MATHWEB connection in $\lambda$-Clam.  

\subsection{Mutual induction module (Brotherston / Smaill)}
The module \texttt{theories/mutual\_induction.mod} is now finished and
contains custom methods designed to apply the mutual induction scheme
used by the MFOTL group at Liverpool.  For a full description of the
induction scheme and the associated methods, please consult section
7.6 of the $\lambda$-Clam manual.

\subsection{Rewriting fix (Maclean)}
The main rewriting predicate has been changed to take proper account
of the direction of rewrite rules.

\section{Contact}
The main $\lambda$-Clam developer is James Brotherston, email:
jjb@dai.ed.ac.uk or telephone: +44 (0)131-650-2721.

\end{document}
