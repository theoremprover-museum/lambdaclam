\documentclass[11pt]{article}

\begin{document}

\title{$\lambda$-Clam v3.1.0: Release Note} 

\author{James Brotherston}

\date{September 7, 2001}

\maketitle

\section{New features with this release}

\subsection{ProofGeneral interface}

$\lambda$-Clam is now distributed with a version of the generic
ProofGeneral interface.  Opening .lcm files in XEmacs automatically
invokes ProofGeneral; the top buffer contains the .lcm script file
with your Lambda-Clam commands and the lower buffer displays the
LambdaClam response as you step through the commands using the toolbar
buttons.  An example file (tutorial.lcm) is
included in lambda-clam-teyjus/ProofGeneral/lclam/ to help you get started. \\

\noindent\textbf{NB.} In order to use ProofGeneral, you need to add the
following line to your .emacs file: 

\noindent\texttt{(load "LCLAM\_PATH/lambda-clam-teyjus/ProofGeneral/generic/proof-site.el")} 

\noindent where \texttt{LCLAM\_PATH} is the path to your lambda-clam-teyjus 
directory.

\subsection{Spypoint facility}

The planner core now contains a spypoint mechanism so that you can
tell Lambda-Clam to interrupt planning immediately after the
successful application of a given method.  Once planning has been
interrupted, you have all the options normally available in
step-by-step (interactive) mode:
\begin{itemize}
\item \texttt{continue}
\item \texttt{backtrack}
\item \texttt{abandon}
\item \texttt{plan\_node [node]}
\item \texttt{try [method]}
\end{itemize}
\noindent Set spypoints with the command \texttt{set\_spypoint [method]} 
and remove them with \texttt{remove\_spypoint [method]}.

\subsection{Improved output functionality}

Goals, addresses and successful method applications are now explicitly
marked in the $\lambda$-Clam output.  The planner can now also be
run in \emph{silent mode}, which suppresses output from failed method
applications and backtracking attempts.  We suggest that you run
$\lambda$-Clam in silent mode most of the time and use normal mode only 
for debugging purposes. \\

\noindent Switch between silent and normal output modes with the command
\texttt{silent\_output [on/off]}.

\section{Future development}

This section details some of the ideas currently under consideration
for future developments.  Feedback on these ideas is widely encouraged.

\begin{enumerate}
\item Improvements to the critics suite to bring them closer to the
  performance standards achieved previously in CLAM.
\item Fix for the known problems with embedding unifications.
\item Integration of the existing code (due to Ewen Denney) for theory
  compilation from `pretty' syntax to $\lambda$-Clam's internal
  syntax.  This could also be integrated with ProofGeneral.
\item Improved support for case-splitting.
\item Implementation of alternative planning strategies
  (breadth-first, iterative deepening).
\end{enumerate}

\section{Contact}
The main $\lambda$-Clam developer is James Brotherston, email:
jjb@dcs.ed.ac.uk or telephone: 0131-650-2721.

\end{document}
