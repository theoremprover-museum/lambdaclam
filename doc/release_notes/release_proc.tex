\documentclass[11pt]{article}

\begin{document}

\title{$\lambda$-Clam: Release Procedure} 

\author{James Brotherston}

\date{August 23, 2002}

\maketitle

\noindent Now you too can release an official version of $\lambda$-Clam in just
10 easy steps:

\begin{enumerate}
\item Write a release note detailing the changes made since the last
  release.  Store it with the other release notes in the CVS
  repository under {\tt lambda-clam-teyjus/doc/release\_notes}.  You
  will need to add the file(s) to the CVS repository using the command
  {\tt cvs add <filename>}.
\item The way the version number changes depends on the number and
  type of changes made since the last release.  At time of writing,
  the current version is v4.0.0.  If the changes made are mostly bug
  fixes and minor improvements, the new version would be v4.0.1; if
  the release incorporated new features or more major changes then the
  new version would be v4.1.0.  Generally the first number only
  increases when wholesale changes have been made or, as has happened
  in the past, the whole system has been rewritten in a different language.
\item Ensure all the benchmarks run OK.  To compile the benchmarks, go
  to {\tt lambda-clam-teyjus/src} and type {\tt make allclean}
  followed by {\tt make benchmarks}.  One or two of the files being
  compiled might report ``Access to unmapped memory'' or something
  similar, in which case you should comment / uncomment (as
  applicable) the `dummy predicate' in the top file being compiled.
  
  To run the benchmarks, go to {\tt lambda-clam-teyjus/bin} and type
  {\tt . benchmark -timeout 600}. This will run the benchmark set with
  a time limit per theorem of 600 seconds (10 minutes), which should
  be adequate for the theorems in the benchmark set at time of
  writing.  The whole thing will probably take an hour or two on a
  local machine; once it's finished, a summary report will be in {\tt
    lambda-clam-teyjus/tests/report.tex}.  Convert this to .ps if you
  want and check that all the benchmarks plan successfully --- the
  ones that don't will have an N next to their time measurement.  The
  only one that should have the N is {\tt falsearith}, although
  currently the theorems all\_list\_cv and allList\_and\_dist are
  failing too (running out of memory at the ripple analysis stage).
\item Ensure the manual is up to date.  The $\lambda$-Clam manual
  files are in {\tt lambda-clam-teyjus/doc}.
\item Make sure that all the changes to CVS have been committed
  (including the updated manual and release note). Committing files is
  just a matter of {\tt cvs commit <filenames>}.
\item From {\tt lambda-clam-teyjus}, tag the new version in the CVS
  repository using the command {\tt cvs tag lclamv4-x-y} with {\tt
    x} and {\tt y} replaced by the appropriate numbers for the new
  version (see item 2).
\item From your home directory, make a tarball of the CVS distribution
  of $\lambda$-Clam using the command: \\ {\tt tar cf
    lambda-clam4.x.y.tar lambda-clam-teyjus} \\ \noindent and then zip
  it using {\tt gzip lambda-clam4.x.y.tar}.
\item The $\lambda$-Clam webpage is accessible at: \\ {\tt
    /usr/local/reason/htdocs/software/systems/lambda-clam} \\  From here,
  put PS copies of the release note and of the manual in the directory
  {\tt docs}, and the gzipped $\lambda$-Clam tarball in {\tt src}.
  Then update the main page ({\tt index.html}) to point to all these things.
\item Send an e-mail to {\tt dreamers@dai} announcing the new release.
\item Leave for the rest of the day.

\end{enumerate}


\end{document}
